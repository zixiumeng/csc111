\documentclass[11pt]{article}
\usepackage{amsmath}
\usepackage[utf8]{inputenc}
\usepackage[margin=0.75in]{geometry}

\title{CSC111 Assignment 3: Graphs, Recommender Systems, and Clustering}
\author{Zixiu Meng, Jingyang Yu}
\date{\today}

\begin{document}
    \maketitle

    \section*{Part 1: The book review graph and simple recommendations}

    \begin{enumerate}

        \item[1.]
        Complete this part in the provided \texttt{a3\_part1.py} starter file.
        Do \textbf{not} include your solution in this file.

        \item[2.]
        1. the first for loop has a running time of $\Theta{(m)}$ because the loop iterate m times and the opening of csv file and each iteration of the loop body is a constant time operation.\\
        2. the second for loop has a running time of $\Theta{(n)}$ because the loop iterate n times. though the loop body contains two steps, addedge and addvertex are both constant time operation. \\
        3. hence the overall time complexity of the implementation is $\Theta{(m+n)}$

        \item[3.]
        Complete this part in the provided \texttt{a3\_part1.py} starter file.
        Do \textbf{not} include your solution in this file.

        \item[4.]
        Complete this part in the provided \texttt{a3\_part1.py} starter file.
        Do \textbf{not} include your solution in this file.

    \end{enumerate}

    \section*{Part 2: Weighted graphs, recommendations, review prediction}

    Complete this part in the provided \texttt{a3\_part2\_recommendations.py} and \texttt{a3\_part2\_predictions.py} starter files.
    Do \textbf{not} include your solution in this file.

    \newpage

    \section*{Part 3: Finding book clusters}

    \begin{enumerate}

        \item[1.]
        Complete this part in the provided \texttt{a3\_part3.py} starter file.
        Do \textbf{not} include your solution in this file.

        \item[2.]
        Complete this part in the provided \texttt{a3\_part3.py} starter file.
        Do \textbf{not} include your solution in this file.

        \item[3.]

        \begin{enumerate}
            \item[(a)]
            1.the outer for loop iterates $m_1$ times. \\
            2.the inner for loop iterates $m_2$ times. \\
            3.the inner loop body is a constant time operation. \\
            4.the overall time complexity is $\Theta{(m_1 m_2)}$

            \item[(b)]
            1. "if score is greater than best" part is a constant time operation. \\
            2. "cross cluster weight(graph, c1, c2)" has a running time of $\Theta{(c_1 c_2)}$ \\
            3. since cluster1 has fixed size k, the running time of "cross cluster weight(graph, c1, c2)" is $\Theta{(c_2)}$\\
            4. the total running time is $\sum_{i=1}^{j}ac_{2i}$ where j is the number of iteration\\
            5. the sum of all cluster size is n which means  $\sum_{i=1}^{j}c_{2i}$ = n. Therefore, the total running time is kn.\\
            6. k can be written as n/p for some number p.
            5.hence the running time for inner loop is n*n/p, which$\mathcal{O}(n^2)$


            \item[(c)]
            1. the set operation takes n times\\
            Inside the for loop:\\
            2. the first two reassignment takes constant time.\\
            3. as mentioned previously, the inner loop has an upper asymptote of $n^2$ \\
            4. set.update takes the proportion of size of best\_c1, which means it takes at most n steps.\\
            5. list.remove takes at most n times\\
            6. the loop body of outer loop takes $n^2 + 2n + 1$times, which is $\mathcal{O}(n^2)$\\
            7. the outer loop iterates $n-k$ times \\
            8. the overall upper asymptote for this algorithm is $\mathcal{O}(n^2(n-k)+n)$, which is $\mathcal{O}(n^2(n-k))$

            \item[(d)]
            \emph{Not to be handed in.}
        \end{enumerate}

    \end{enumerate}
\end{document}
