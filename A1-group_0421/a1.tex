\documentclass[fontsize=11pt]{article}
\usepackage{amsmath}
\usepackage[utf8]{inputenc}
\usepackage[margin=0.75in]{geometry}

\title{CSC111 Assignment 1: Linked Lists}
\author{Zixiu Meng, Jingyang Yu}
\date{\today}

\begin{document}
    \maketitle

    \section*{Part 1: Faster Searching in Linked Lists}

    \begin{enumerate}

        \item[1.]
        Complete this part in the provided \texttt{a1\_part1.py} starter file.
        Do \textbf{not} include your solution in this file.

        \item[2.]
        Complete this part in the provided \texttt{a1\_part1\_test.py} starter file.
        Do \textbf{not} include your solution in this file.

        \item[3.]
        \begin{enumerate}

            \item[(a)]
            Let n $\in Z^+$ and $n>2$, the list is 0,1,2...n-1\\
            Let m $\in Z^+$\\
            The for loop iterate m times\\
            In the iteration:
            \begin{itemize}
                \item The first line takes 1 step
                \item The while loop iterates n times, each iteration takes 1 step
                \item In total, the while loop takes n steps
            \end{itemize}
            Therefore, the total running time is m(n+1), which is $\Theta(mn)$ steps.


            \item[(b)]
            \textbf{Heuristic 1 (move to front)}\\
            Let n $\in Z^+$ and $n>2$, the list is 0,1,2...n-1\\
            Let m $\in Z^+$\\
            The for loop iterates m times\\
            The first search: The first two line takes 1 step.\\
            the while loop iterate n times, each iteration has one step, then (n-1) move to the front of the list which also takes one step.\\
            The second search: The first two line takes 1 step.\\
            since (n-1) is the first, the while loop only take 1 step and the rest of code takes 1 step.In total, it is 3 steps but we can consider it as one step.\\
            All other search: same to the second research since the position of (n-1) does not change. \\
            In total, the running time will be (1+n+1)+(m-1), which is $\Theta({m+n})$

            \textbf{Heuristic 2 (swap)}\\
            case1 $m<n$\\
            Let n $\in Z^+$ and $n>2$, the list is 0,1,2...n-1\\
            Let m $\in Z^+$\\
            The for loop iterates m times\\
            The first search: the first two lines take one step.\\
            The while loop iterates n times and each iteration takes 1 step.\\
            (n-1) swap its position with (n-2) and it takes 1 step\\
            The second search: the first two lines take one step\\
            The while loop iterates n-1 times and each iteration takes 1 step.\\
            (n-1) swap its position with (n-3) and it takes 1 step\\
            the mth search: the first two lines take one step\\
            the while loop iterates n-m times and each iteration takes 1 step.\\
            the rest code takes 1 step\\
            Therefore, the total running time is $\sum_{i=1}^{m} (1+n-i+1+1) = \frac{(2n-1-m)m}{2}+3m$, which is $\Theta({mn-m^{2}})$, which is $\Theta(mn)$

            case2 $m>=n$
            Let n $\in Z^+$, the list is 0,1,2...n-1\\
            Let m $\in Z^+$\\
            Similar to what we analyse before, in the kth $(k<n)$ search, the first two lines take 1 step, the while loop iterates (n-k+1) times, the rest code takes 1 step. \\
            After (n-1) become the first element in the linked list, the while loop always iterates 1 step and the rest code takes constant time.
            Therefore, the total running time is $\sum_{i=1}^{n} (1+n-i+1+1)$+(m-n) = $\frac{(n+1)n}{2}+m+n$, which is $\Theta(n^{2}+m)$

            \textbf{Heuristic 3 (count)}
            Let n $\in Z^+$, the list is 0,1,2...n-1 and all counts are 0\\
            Let m $\in Z^+$\\
            the for loop iterates m times\\
            In the first search, the first three lines takes 1 step, \\
            the while loop iterate n times and each iteration takes 1 step.
            n-1 is moved to the front of the linked list which also takes 1 step.\\
            In the rest search, the first three lines takes 1 step, \\
            the while loop iterate 1 times and the iteration takes 1 step.
            other code takes 1 step.\\
            Therefore, the total running time is (1+n+1)+3(m-1), which is $\Theta(n+m)$


        \end{enumerate}

        \item[4.]
        Let lst be a list of number [1, 2, 3, 4...n], with length of n.
        Let m be a sequence of (n, n-1, n-2, n-3...1, 1, 1...) with length m.\\
        Assume $m>n$\\
        part1: when searching for (n, n-1, n-2...1)
        By comparing the code for Heuristic1 and Heuristic2, the only difference between them is the while loop because all other code takes constant time.\\
        The first search analysis for movetofirst: the while loop iterate n times to reach n, each iteration is 1 step. Then, the linked list become (n, 1, 2...n-1).\\
        The second search analysis movetofirst: the while loop iterate n times to reach n-1, each iteration is 1 step. Then, the linked list become (n-1, n, 1, 2...n-2).\\
        To conclude, for Heuristic1, when searching for k, the while loop always iterates n times. Therefore, the total running time for part 1 is T1 = $n^2$\\
        The first search analysis for swap: the while loop iterate n times to reach n, each iteration is 1 step. Then, the linked list becomes (1, 2,3...n, n-1)\\
        The second search analysis for swap: the while loop iterate n times to reach n-1, each iteration is 1 step. Then, the linked list becomes (1, 2,3...n-1, n)
        The third search analysis for swap: the while loop iterate n-2 times to reach n-2, each iteration is 1 step. Then, the linked list becomes (1, 2,3...n-2, n-3, n-1, n)
        The fourth search analysis for swap: the while loop iterate n-2 times to reach n-3, each iteration is 1 step. Then, the linked list becomes (1, 2, 3...n-3, n-2, n-1, n)
        To conclude, after each two searches, the list will become the original order. The total running time when n is even is T2 = $\sum_{i=1}^{n/2}(n-2i+2)$=n+$\frac{n^{2}}{2}$.\\
        the total running time when n is odd is T2=$\frac{n^2}{2}+n-\frac{1}{2}$
        part2 when searching for (1, 1, 1, 1...), the linked list already becomes (1, 2, 3...n). For both heuristic, the running time is constant.
        IN conclusion T1-T2 = $\frac{n^2}{2}-n$ (n is even) or $\frac{n^2}{2}-n+\frac{1}{2}$ (n is odd), which is $\Theta(n^2)$, as needed
    \end{enumerate}

    \section*{Part 2: Linked List Visualization}
    Complete this part in the provided \texttt{a1\_part2.py} starter file.
    Do \textbf{not} include your solution in this file.

\end{document}
